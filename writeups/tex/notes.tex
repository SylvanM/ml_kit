\documentclass[12pt, letterpaper]{article}

\usepackage{enumitem}
\usepackage{amsmath}
\usepackage{graphicx}
\usepackage[margin=1in]{geometry}
\usepackage{cancel}
\usepackage{amssymb}
\usepackage{amsfonts}
\usepackage{amstext}
\usepackage{amsthm}
\usepackage{xcolor}
\usepackage{titlesec}
\usepackage{pgfplots}
\usepackage{mdframed}
\usepackage{nicefrac}
\usepackage{dsfont}
\usepackage{tikz}
\usetikzlibrary{trees}
\usepackage{mathdots}
\usepackage{accents}
\usepackage{mathtools}
\usepackage{bbm}

\usepackage{import}

\usepackage[T1]{fontenc}
\usepackage[utf8]{inputenc}
\usepackage{lmodern}
\usepackage[hidelinks]{hyperref}
\usepackage[T1]{fontenc}
\usepackage[utf8]{inputenc}

\usepackage[english]{babel}
\usepackage{csquotes}

\usepackage{xcolor}

\usepackage[notes,backend=biber]{biblatex-chicago}

% theorem environments
\newtheorem{theorem}{Theorem}
\newtheorem{lemma}{Lemma}
\newtheorem{corollary}{Corollary}

\theoremstyle{definition}
\newtheorem{definition}{Definition}
\newtheorem{example}{Example}

\theoremstyle{remark}
\newtheorem*{claim}{Claim}
\newtheorem*{remark}{Remark}
\newtheorem*{note}{Note}

\setlength{\textwidth}{6.0in}
\setlength{\oddsidemargin}{0in}
\setlength{\evensidemargin}{0in}
\setlength{\topmargin}{-0.5in}
\setlength{\headheight}{0.25in}
\setlength{\headsep}{0.25in}
\setlength{\textheight}{8.5in}
\setlength{\footskip}{20pt}

\setlength{\topskip}{0in}

\setcounter{secnumdepth}{0}

\setlength{\parindent}{0in}	
\setlength{\parskip}{0.1in}

\newcommand{\vect}[1]{\vec{\mathbf{#1}}}
\newcommand{\sectionbreak}{\clearpage}
\newcommand{\R}{\mathbb{R}}
\newcommand{\Z}{\mathbb{Z}}
\newcommand{\N}{\mathbb{N}}
\renewcommand{\P}{\mathbb{P}}
\newcommand{\F}{\mathbb{F}}
\newcommand{\modop}{\;\text{mod}\;}
\renewcommand{\mod}[1]{\;(\text{mod}\;#1)}
\newcommand{\Aut}{\text{Aut}}
\newcommand{\id}{\text{id}}
\newcommand{\spn}{\;\text{span}}

\newcommand{\ftntmk}{\textcolor{blue}{\footnotemark}}
\newcommand{\black}[1]{\textcolor{black}{#1}}
\newcommand{\ubcolor}[2]{\color{#1}{\underbrace{\color{black}{#2}}}}
\newcommand*\circled[1]{\tikz[baseline=(char.base)]{
            \node[shape=circle,draw,inner sep=2pt] (char) {#1};}}

\newcommand{\vol}[1]{\text{vol}}

\newcommand{\verteq}{\rotatebox{90}{$\,=$}}
\newcommand{\equalto}[2]{\underset{\scriptstyle\overset{\mkern4mu\verteq}{#2}}{#1}}

\newcommand{\rk}{\text{rk}\,}
\newcommand{\Col}{\text{Col}\,}
\newcommand{\C}{\mathbb{C}}
\newcommand{\Q}{\mathbb{Q}}
\newcommand{\actson}{\curvearrowright}

\newcommand{\E}{\mathbb{E}}

\renewcommand{\mapsto}{\longmapsto}

% James - Parentheses, brackets, etc.
\newcommand{\ignore}[1]{}  % from Charles
\newcommand{\parens}[1]{\ensuremath{\left( #1 \right)}}
\newcommand{\bracks}[1]{\ensuremath{\left[ #1 \right]}}
\newcommand{\braces}[1]{\ensuremath{\left\{ #1 \right\}}}
\newcommand{\angbrs}[1]{\ensuremath{\langle #1 \rangle}}
\newcommand{\set}[1]{\braces{#1}}
\newcommand{\powset}[1]{\mathcal{P}\parens{#1}}
\newcommand{\vspan}[1]{\angbrs{#1}}  % 4330: Span of a set of vectors

\newcommand{\floor}[1]{\left\lfloor #1 \right\rfloor}
\newcommand{\ceil}[1]{\left\lceil #1 \right\rceil}
\newcommand{\verts}[1]{\left\lvert #1 \right\rvert} % | #1 |
\newcommand{\Verts}[1]{\left\lVert #1 \right\rVert} % || #1 ||
\newcommand{\abs}[1]{\verts{#1}}
\newcommand{\size}[1]{\verts{#1}}
\newcommand{\norm}[1]{\Verts{#1}}

\newcommand{\eps}{\varepsilon}
\newcommand{\vphi}{\varphi}
\renewcommand{\Re}{\mathrm{Re}}
\renewcommand{\Im}{\mathrm{Im}}

\newcommand{\bmat}[1]{\begin{bmatrix} #1 \end{bmatrix}}
\newcommand{\pmat}[1]{\begin{pmatrix} #1 \end{pmatrix}}

\title{Implementing Gradient Descent}
\author{Owen Wetherbee, Ethan Ma, Sylvan Martin}
\date{}

\bibliography{bibliography}

% Body

\pgfplotsset{compat=1.16}
\begin{document}

\maketitle

\begin{center}
    \textbf{Working out Backpropogation}
\end{center}

\subsection{Neural Network Structure}

In this meeting, we went over the math behind neural networks: feed-forwarding,
derivatives, and backpropagation. This document contains what we thought you need to know 
for implementing back-propagation.

Say that we have a feed-forward neural network consisting of $L$ layers, where 
layer $L$ is the output layer, and layer $0$ is the input layer. Let $\vect{a}^{(\ell)}$ represent the activations 
in the $\ell$-th layer of the network. So if the input to our network is the vector $\vect{x}$, then $\vect{a}^{(0)} = \vect{x}$.
For the purposes of this writeup, vectors are 1-indexed, as opposed to in code where they are 0-indexed.

Say that layer $\ell$ has $n_\ell$ neurons.

Let $w_{ij}^{(\ell)}$ represent the weight on the edge from the $j$-th node in layer $\ell - 1$ to the $i$-th node in layer $\ell$.
Let $W^{(\ell)}$ be the matrix defined by 
\[
    W^{(\ell)} = \left[
        \begin{array}{cccc}
            w_{11}^{(\ell)} & w_{12}^{(\ell)} & \cdots & w_{1n_{\ell - 1}}^{(\ell)} \\
            w_{21}^{(\ell)} & w_{22}^{(\ell)} & \cdots & w_{2n_{\ell - 1}}^{(\ell)} \\
            \vdots          & \vdots          & \ddots & \vdots \\
            w_{n_\ell 1}^{(\ell)} & w_{n_\ell 2}^{(\ell)} & \cdots & w_{n_\ell n_{\ell - 1}}^{(\ell)}
        \end{array}
    \right]
\]
Viewed as a linear transformation, this is $W^{(\ell)}: \R^{n_{\ell - 1}} \to \R^{n_\ell}$, and so its dimension is $n_{\ell} \times n_{\ell - 1}$.

Let $b_i^{(\ell)}$ be the bias associated with the $i$-th node of layer $\ell$.
Each layer of the network has a ``squishification function'' written as $\sigma^{(\ell)}$, so computing the activation $a_i^{(\ell)}$ can 
be written as 
\[
    a_i^{(\ell)} = \sigma^{(\ell)} \left(
        z_i^{(\ell)}
    \right) 
\]
where we let 
\[
    z_i^{(\ell)} = b_i^{(\ell)} + \sum_{j = 1}^{n_{\ell - 1}} w_{ij} \, a_j^{(\ell - 1)}
\]
We can also write this more succinctly as  
\[
    \vect{a}^{(\ell)} = \sigma\left( \vect z^{(\ell)} \right)
\]
where \[
    \vect z^{(\ell)} = W^{(\ell)} \vect{a}^{(\ell - 1)} + \vect{b}^{(\ell)}
\]
and where $\sigma(\vect x)$ is applied to each element of $x$.

\subsection{Cost Gradients}

For now, we'll be using squared loss. If for training sample $1$ we desire the output layer to have value $\vect y$, 
\[
    C_1 = \| \vect{a^{(\ell)}} - \vect{y} \|_2^2 = \sum_{i = 1}^{n_\ell} (a_i^{(\ell)} - y_i)^2
\]
The overall cost for the network over all $N$ training samples will be the average of all costs, so 
\[
    C = \frac{1}{N}\sum_{k = 1}^{N} C_k
\]
We wish to compute the gradient, $\nabla C$, of the loss function, so that we can take a step in the ``downwards'' direction
along the surface formed by the graph of $C$ in order to find a minimum of $C$. Since we only care about the direction 
the gradient is pointing and not the magnitude, the factor of $\frac{1}{N}$ in front can be ignored.\ftntmk{} So, we care about 
computing 
\footnotetext{From here on out, for two vectors $\vect v$ and $\vect u$, $\vect v \approx \vect u$ will mean that the two 
vectors are pointing in the same direction, but may not have the same magnitude. More formally,
\[
    \vect v \approx \vect u \iff \frac{\vect v}{\|\vect v\|} = \frac{\vect u}{\|\vect u\|}
\]}
\[
    \nabla C \approx \nabla C_0 + \nabla C_1 + \cdots + \nabla C_N
\]
For explanation purposes, we'll go through computing $\nabla C_0$ for a label $\vect y$, with input $\vect x = \vect a^{(0)}$.
The gradient is 
\[
    \nabla C_0 = \left[
        \begin{array}{c}
            \nicefrac{\partial C_0}{\partial w_{00}^{(1)}} \\
            \vdots  \\
            \nicefrac{\partial C_0}{\partial w_{ij}^{(1)}} \\
            \vdots \\
            \nicefrac{\partial C_0}{\partial w_{n_1 n_0}^{(1)}} \\
            \vdots \\
            \nicefrac{\partial C_0}{\partial b_{i}^{(1)}} \\
            \vdots \\
        \end{array}
    \right]
\]
Where the dimension of this vector is the number of total parameters (weights and biases) of our network. It's components
each reflect how sensitive the overall cost is to a small change in one of the parameters, so we want to take a step 
in the most efficient direction to decrease the cost.

\subsubsection{Computing Partial Derivatives}

Using the chain rule, we can compute the derivative with respect to one of the weights in layer $\ell$.
\[
    \frac{\partial C_0}{\partial w_{ij}^{(\ell)}} = \frac{\partial z_i^{(\ell)}}{\partial w_{ij}^{(\ell)}} \cdot 
                                                    \frac{\partial a_i^{(\ell)}}{\partial z_i^{(\ell)}} \cdot 
                                                    \frac{\partial C_0}{\partial a_i^{(\ell)}}
\]
In the same manor we can compute the derivative with respect to one of the biases.
\[
    \frac{\partial C_0}{\partial b_i^{(\ell)}} = \frac{\partial z_i^{(\ell)}}{\partial b_i^{(\ell)}} \cdot 
                                                    \frac{\partial a_i^{(\ell)}}{\partial z_i^{(\ell)}} \cdot 
                                                    \frac{\partial C_0}{\partial a_i^{(\ell)}}
\]
We can actually simplify these computations quite a lot. Using the formula for $z_i^{(\ell)}$, we know 
\[
    z_i^{(\ell)} = b_i^{(\ell)} + \left(\sum_{j = 1}^{n_{\ell - 1}} w_{ij}^{(\ell)} a_j^{(\ell - 1)}\right)
    \implies \frac{\partial z_i^{(\ell)}}{\partial w_{ij}^{(\ell)}} = a_j^{(\ell - 1)}
\]
When taking the derivative with respect to bias, this becomes much simpler.
\[
    z_i^{(\ell)} = b_i^{(\ell)} + \left(\sum_{j = 1}^{n_{\ell - 1}} w_{ij}^{(\ell)} a_j^{(\ell - 1)}\right)
    \implies \frac{\partial z_i^{(\ell)}}{\partial b_{i}^{(\ell)}} = 1
\]
Also, because $a_i^{(\ell)} = \sigma^{(\ell)}(z_i^{(\ell)})$, $\frac{\partial a_i^{(\ell)}}{\partial z_i^{(\ell)}} = \dot{\sigma}^{(\ell)}(z_i^{(\ell)})$
where $\dot\sigma$ is the derivative of $\sigma$. Together, this means 
\begin{align*}
    \frac{\partial C_0}{\partial w_{ij}^{(\ell)}} &= a_j^{(\ell - 1)} \dot\sigma^{(\ell)}(z_i^{(\ell)})\cdot\frac{\partial C_0}{\partial a_i^{(\ell)}} \\
    \frac{\partial C_0}{\partial b_{i}^{(\ell)}} &= \dot\sigma^{(\ell)}(z_i^{(\ell)})\cdot\frac{\partial C_0}{\partial a_i^{(\ell)}} \\
\end{align*}
Let's use matrix notation to clean this up a bit. Let $\frac{\partial C_0}{\partial W^{(\ell)}}$ represent the matrix whose $(i, j)$-th entry 
is $\frac{\partial C_0}{\partial w_{ij}^{(\ell)}}$. Likewise, $\frac{\partial C_0}{\partial {\vect b}^{(\ell)}}$ is the vector whose $i$-th 
entry is $\frac{\partial C_0}{\partial b_{i}^{(\ell)}}$. Now, we can write 
\[
    \frac{\partial C_0}{\partial \vect{b}^{(\ell)}} = \dot\sigma^{(\ell)}(\vect{z}^{(\ell)}) \odot \frac{\partial C_0}{\partial \vect{a}^{(\ell)}} \;\;\; \text{and} \;\;\; \frac{\partial C_0}{\partial W^{(\ell)}} = \frac{\partial C_0}{\partial \vect{b}^{(\ell)}} \left(\vect{a}^{(\ell - 1)}\right)^\top
\]
Where $\odot$ represents the point-wise \textit{Hadamard product}.


This leaves the question of how to compute the derivative of $C_0$ with respect to $a_i$ for each layer. Notice that if $\ell = L$ (we 
are in the last layer) this is actually quite straightforward. Using the definition of cost,
\[
    C_0 = \sum_{i = 1}^{n_L} (a_i^{(L)} - y_i)^2
\]
we can easily compute the derivative
\[
    \frac{\partial C_0}{\partial a_i^{(L)}} = 2(a_i^{(L)} - y_i) \;\;\; \text{or} \;\;\; \frac{\partial C_0}{\partial \vect{a}^{(L)}} = 2(\vect{a}^{(L)} - \vect{y})
\]
However, if we try to find an expression for the same derivative but in a previous layer, we find 
\[
    \frac{\partial C_0}{\partial a_k^{(\ell - 1)}} = \sum_{j = 1}^{n_\ell} \frac{\partial z_j^{(\ell)}}{\partial a_i^{(\ell - 1)}} \cdot 
                                                                             \frac{\partial a_j^{(\ell)}}{\partial z_j^{(\ell)}} \cdot 
                                                                             \frac{\partial C_0}{\partial a_j^{(\ell)}}
                                                   = \sum_{j = 1}^{n_\ell}  w_{jk}^{(\ell)} \dot\sigma^{(\ell)}(z_j^{(\ell)}) \cdot \frac{\partial C_0}{\partial a_j^{(\ell)}}
\]
In matrix notation, this is 
\[
    \frac{\partial C_0}{\partial \vect{a}^{(\ell - 1)}} = {W^{(\ell)}}^\top \left(\dot\sigma^{(\ell)}\left(\vect{z}^{(\ell)}\right) \odot \frac{\partial C_0}{\partial \vect{a}^{(\ell)}}\right)
\]
Notice this formula is recursive! To compute it efficiently, we can use a dynamic programming style of approach. This 
gives us the following natural algorithm for computing $\nabla C_0$.

\begin{center}
    \textbf{The Backpropogation Algorithm}
\end{center}

\emph{(Base case of the DP table.)} Start by computing all $\nicefrac{\partial C_0}{\partial a_i^{(L)}} = 2(a_i^{(L)} - y_i)$ for $1 \leq i \leq n_L$.
With this done, we can also calculate all 
\[
    \frac{\partial C_0}{\partial w_{ij}^{(L)}} = a_i^{(L - 1)} \dot\sigma^{(\ell)}(z_i^{(L)}) \frac{\partial C_0}{\partial a_i^{(L)}} \;\;\; \text{and} \;\;\; \frac{\partial C_0}{\partial b_i^{(L)}} = \dot\sigma^{(\ell)}(z_i^{(\ell)}) \frac{\partial C_0}{\partial a_i^{(L)}}
\]
for the last layer $L$. In matrix form, this means computing 
\begin{align*}
    \frac{\partial C_0}{\partial \vect{a}^{(L)}} &= 2\left(\vect{a}^{(L)} - \vect y\right) \\
    \frac{\partial C_0}{\partial \vect{b}^{(L)}} &= \dot\sigma^{(L)}(\vect{z}^{(L)}) \odot \frac{\partial C_0}{\partial \vect{a}^{(L)}} \\
    \frac{\partial C_0}{\partial W^{(L)}}        &= \frac{\partial C_0}{\partial \vect{b}^{(L)}} \left(\vect{a}^{(L - 1)}\right)^\top
\end{align*}

\emph{(Recursive case of DP table)} Now, iterating $\ell$ from $L - 1$ down to $1$, compute for all $1 \leq i \leq n_\ell$ the derivatives 
\[
    \frac{\partial C_0}{\partial a_i^{(\ell)}} = \sum_{j = 1}^{n_{(\ell + 1)}} w_{ij}^{(\ell)} \dot\sigma^{(\ell + 1)}(z_j^{(\ell + 1)}) \frac{\partial C_0}{\partial a_j^{(\ell + 1)}}
\]
Again, in matrix form, this is computing 
\[
    \frac{\partial C_0}{\partial \vect{a}^{(\ell)}} = W^{(\ell)} \left(\dot\sigma^{(\ell + 1)}\left(\vect{z^{(\ell)}}\right) \odot \frac{\partial C_0}{\partial \vect{a}^{(\ell + 1)}}\right)
\]
Once these have been computed, one can directly compute
\[
    \frac{\partial C_0}{\partial w_{ij}^{(\ell)}} = a_j^{(\ell - 1)} \dot\sigma^{(\ell)}(z_i^{(\ell)}) \frac{\partial C_0}{\partial a_i^{(\ell)}} \;\;\; \text{and} \;\;\; \frac{\partial C_0}{\partial b_i^{(\ell)}} = \dot\sigma^{(\ell)}(z_i^{(\ell)}) \frac{\partial C_0}{\partial a_i^{(\ell)}}
\]
which is 
\begin{align*}
    \frac{\partial C_0}{\partial \vect{b}^{(\ell)}} &= \dot\sigma^{(\ell)}(\vect{z}^{(\ell)}) \odot \frac{\partial C_0}{\partial \vect{a}^{(\ell)}} \\
    \frac{\partial C_0}{\partial W^{(\ell)}}        &= \frac{\partial C_0}{\partial \vect{b}^{(\ell)}} \left(\vect{a}^{(\ell - 1)}\right)^\top
\end{align*}

And that's it! This gives everything you need to fully compute $\nabla C_0$.

\begin{center}
    \textbf{\emph{Stochastic} Gradient Descent}
\end{center}

Fully computing $\nabla C \approx \nabla C_0, \ldots, \nabla C_N$ is very costly, as that's a lot of gradients to compute. So instead 
of recomputing $\nabla C$ and taking a step in the $-\nabla C$ direction every time, we first start by randomly partitioning 
our training set into $B$ ``batches.'' We'll say that $C_{k, b}$ is the cost of the network on the $b$-th sample of the $k$-th 
batch of our training set, and $\nabla C_b \approx \nabla C_{1, b} + \cdots + \nabla C_{N/B, b}$ for $1 \leq b \leq B$. 
At each step of gradient descent, we iterate over $1 \leq b \leq B$, taking a step in the $-\nabla C_b$ direction. 
We repeat this iteration until some other stopping condition.

\end{document}

