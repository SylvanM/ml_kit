\documentclass[12pt, letterpaper]{article}

\usepackage{enumitem}
\usepackage{amsmath}
\usepackage{graphicx}
\usepackage[margin=1in]{geometry}
\usepackage{cancel}
\usepackage{amssymb}
\usepackage{amsfonts}
\usepackage{amstext}
\usepackage{amsthm}
\usepackage{xcolor}
\usepackage{titlesec}
\usepackage{pgfplots}
\usepackage{mdframed}
\usepackage{nicefrac}
\usepackage{dsfont}
\usepackage{tikz}
\usetikzlibrary{trees}
\usepackage{mathdots}
\usepackage{accents}
\usepackage{mathtools}
\usepackage{bbm}
\usepackage{caption}
\usepackage{float}

\usepackage{import}

\usepackage[T1]{fontenc}
\usepackage[utf8]{inputenc}
\usepackage{lmodern}
\usepackage[hidelinks]{hyperref}
\usepackage[T1]{fontenc}
\usepackage[utf8]{inputenc}

\usepackage[english]{babel}
\usepackage{csquotes}

\usepackage{xcolor}



\usepackage[notes,backend=biber]{biblatex-chicago}

% theorem environments
\newtheorem{theorem}{Theorem}
\newtheorem{lemma}{Lemma}
\newtheorem{corollary}{Corollary}

\theoremstyle{definition}
\newtheorem{definition}{Definition}
\newtheorem{example}{Example}

\theoremstyle{remark}
\newtheorem*{claim}{Claim}
\newtheorem*{remark}{Remark}
\newtheorem*{note}{Note}

\setlength{\textwidth}{6.0in}
\setlength{\oddsidemargin}{0in}
\setlength{\evensidemargin}{0in}
\setlength{\topmargin}{-0.5in}
\setlength{\headheight}{0.25in}
\setlength{\headsep}{0.25in}
\setlength{\textheight}{8.5in}
\setlength{\footskip}{20pt}

\setlength{\topskip}{0in}

\setcounter{secnumdepth}{0}

\setlength{\parindent}{0in}	
\setlength{\parskip}{0.1in}

\newcommand{\vect}[1]{\vec{\mathbf{#1}}}
% \newcommand{\sectionbreak}{\clearpage}
\newcommand{\R}{\mathbb{R}}
\newcommand{\Z}{\mathbb{Z}}
\newcommand{\N}{\mathbb{N}}
\renewcommand{\P}{\mathbb{P}}
\newcommand{\F}{\mathbb{F}}
\newcommand{\modop}{\;\text{mod}\;}
\renewcommand{\mod}[1]{\;(\text{mod}\;#1)}
\newcommand{\Aut}{\text{Aut}}
\newcommand{\id}{\text{id}}
\newcommand{\spn}{\;\text{span}}

\DeclareMathOperator*{\argmax}{argmax}

\newcommand{\ftntmk}{\textcolor{blue}{\footnotemark}}
\newcommand{\black}[1]{\textcolor{black}{#1}}
\newcommand{\ubcolor}[2]{\color{#1}{\underbrace{\color{black}{#2}}}}
\newcommand*\circled[1]{\tikz[baseline=(char.base)]{
            \node[shape=circle,draw,inner sep=2pt] (char) {#1};}}

\newcommand{\vol}[1]{\text{vol}}

\newcommand{\verteq}{\rotatebox{90}{$\,=$}}
\newcommand{\equalto}[2]{\underset{\scriptstyle\overset{\mkern4mu\verteq}{#2}}{#1}}

\newcommand{\rk}{\text{rk}\,}
\newcommand{\Col}{\text{Col}\,}
\newcommand{\C}{\mathbb{C}}
\newcommand{\Q}{\mathbb{Q}}
\newcommand{\actson}{\curvearrowright}

\newcommand{\E}{\mathbb{E}}

\renewcommand{\mapsto}{\longmapsto}

% James - Parentheses, brackets, etc.
\newcommand{\ignore}[1]{}  % from Charles
\newcommand{\parens}[1]{\ensuremath{\left( #1 \right)}}
\newcommand{\bracks}[1]{\ensuremath{\left[ #1 \right]}}
\newcommand{\braces}[1]{\ensuremath{\left\{ #1 \right\}}}
\newcommand{\angbrs}[1]{\ensuremath{\langle #1 \rangle}}
\newcommand{\set}[1]{\braces{#1}}
\newcommand{\powset}[1]{\mathcal{P}\parens{#1}}
\newcommand{\vspan}[1]{\angbrs{#1}}  % 4330: Span of a set of vectors

\newcommand{\floor}[1]{\left\lfloor #1 \right\rfloor}
\newcommand{\ceil}[1]{\left\lceil #1 \right\rceil}
\newcommand{\verts}[1]{\left\lvert #1 \right\rvert} % | #1 |
\newcommand{\Verts}[1]{\left\lVert #1 \right\rVert} % || #1 ||
\newcommand{\abs}[1]{\verts{#1}}
\newcommand{\size}[1]{\verts{#1}}
\newcommand{\norm}[1]{\Verts{#1}}
\newcommand{\eps}{\varepsilon}
\newcommand{\vphi}{\varphi}
\renewcommand{\Re}{\mathrm{Re}}
\renewcommand{\Im}{\mathrm{Im}}

\newcommand{\bmat}[1]{\begin{bmatrix} #1 \end{bmatrix}}
\newcommand{\pmat}[1]{\begin{pmatrix} #1 \end{pmatrix}}

\bibliography{bibliography}

\title{ML Kit: A Machine Learning Library for Rust}

\author{Owen Wetherbee (ocw6), Ethan Ma (em834), Sylvan Martin (slm338)}
\date{}


% Body

\pgfplotsset{compat=1.16}
\begin{document}
\captionsetup{labelformat=empty,labelsep=none}

\maketitle

\begin{center}
    \textbf{Keywords:} Machine learning, Gradient Descent, PCA, Rust
\end{center}

\textbf{Presentation:} \url{https://youtu.be/dLAc-HP8XBQ}

\subsubsection*{Application Setting}

Rust is a relatively new programming language with a dearth of machine learning infrastructure. We aimed to 
create a library for Rust programmers to make machine learning convenient, much like NumPy or TensorFlow in Python.

\section{Project Description}

Initially, we wanted to create a comprehensive machine learning library for Rust, which would use Rust's safety
and speed to implement many algorithms used in data science. We had the original (lofty) goal of being able to 
run a diffusion model by the end of the semester, or be able to do anything that NumPy/SciKitLearn could do. As 
we began implementation we recognized that we lacked the time and resources to implement the sheer amount of algorithms 
we set out to. So, we shifted our focus to what we believe are some of the core algorithms in the field of 
Machine Learning.

In the end, we created the pure-Rust library \texttt{ml\_kit}, which implements, from scratch, the following:
\begin{itemize}
    \item Neural Network based learning, consisting of
    \begin{itemize}
        \item the basic neural network model with user-defined network shape and activation functions 
        for each layer
        \item functionality for handling large datasets for training and testing
        \item a stochastic gradient descent trainer, with whatever batch size and epochs a user may want
        \item various ``gradient update schedules,'' such as fixed learning rates, time-decay learning rates, AdaGrad, etc.
        \item Convolutional Neural Networks
    \end{itemize}
    \item Principle Component Analysis, consisting of
    \begin{itemize}
        \item an implementation of Singular Value Decomposition (SVD),
        \item using SVD to obtain a $k$-dimensional plane of best fit for a set of points in $\R^n$,
        \item compressing images (or any data) using SVD by truncating low-variance dimensions
    \end{itemize}
\end{itemize}

\subsection{Neural Networks}
Neural network based learning (i.e. deep learning) is a type of supervised learning that involves using black-box models which depend on a large number of tunable parameters to classify complicated inputs.
The models are trained by iteratively evaluating their performance on training examples and updating their parameters accordingly, usually through gradient descent.
We call these black-box models `networks' because they are generally conceptualized as a series of composed functions, or `layers', where each layer linearly maps some input vector (or, more generally, tensor) to a possibly differently-sized output vector, and then applies a simple element-wise non-linear `activation function'.
This sequence of highly inter-connected linear functions interspersed with element-wise non-linearities enables the model to identify complex features only using parameters that have relatively simple and computable gradients, ensuring efficient training.

\subsubsection{Fully-connected neural networks}
The traditional neutral network (NN) is a series of fully-connected layers, meaning each input element of a layer has a parameter through which it can linearly affect each output element.
In particular, the function associated with each layer is of the form \textcolor{blue}{\autocite{Nielsen_2015}}
\begin{equation*}
  \begin{aligned}
    \vect{v}_{\text{out}} = f(\vect{v}_{\text{in}}) = \sigma\left(\mathbf{W}\vect{v}_{\text{in}} + \vect{b}\right) ~,
  \end{aligned}
\end{equation*}
where $\mathbf{W}$ is a matrix of `weight' parameters, $\vect{b}$ is a vector of `bias' parameters, and $\sigma$ is the non-linear activiation function.
The derivative of the output vector with respect to the input vector and these parameters can then be simply calculated as
\begin{equation*}
  \begin{aligned}
    \frac{\partial v_{\text{out}, i}}{\partial W_{i, j}} = v_{\text{in}, j} \cdot \sigma'\left(\vect{x}\right)\hspace{3pt}, \hspace{30pt} \frac{\partial v_{\text{out}, i}}{\partial b_{i}} & = \sigma'\left(\vect{x}\right)\hspace{3pt},\text{ and} \hspace{23pt} \frac{\partial v_{\text{out}, i}}{\partial v_{\text{in}, j}} = W_{i, j} \cdot \sigma'\left(\vect{x}\right) ~,
  \end{aligned}
\end{equation*}
where $\vect{x} = \mathbf{W}\vect{v}_{\text{in}} + \vect{b}$ and all unspecified derivatives are zero.
Thus, the output of the entire neural-network model can be simply computed as the composition of each layer function, and the gradient of the loss with respect to the model parameters (to use for parameter updates) can be determined by iteratively applying the chain rule to the above derivatives.

\begin{center}
 \textbf{Implementation:}
\end{center}

The NeuralNetwork implementation within ML_kit is straightforward and exncapsulates everything needed to build, inspect, and run a feed-forward network.
Initially, a NeuralNet holds three parallel vectors: 

1. Weight Matrices $Vec<Matrix<f64>>$, each matrix represents the weight between two layers. 
2. Bias Vectors, one per non-input layer. Each is stored in a column sized vector of size $m * 1$ where m is the number of neurons in said layer. 
pre-activation vector prior to applying the activation function. 
3. Activation Functions, a vector of activations indicate which activation is to be used upon each layer. Sigmoid, Relu, etc. 

Our NeuralNet construction allows for two ways of instantiation. You can call NeuralNet::new(weights, biases, activations) 
to instantiate a network of user provided parameters. Otherwise, for ease of testing, you can call NeuralNet::from_shape(shape, activations)
to create a zero initialized network and fill with NeuralNet::random_network(shape, activations). 

Once instantiated, our compute_final_layer(input) function incorporates the forward pass logic. Starting with a column vector, for each layer 
$l$ we multiply with weight matrix $W_l$ and add bias $b_l$, and apply the element wise activation. The resulting column is then 
carried into the next layer until we return the output vector. We also incorporate functions such as compute_raw_layers() and compute_raw_and_full_layers(). 

We also implemented a paramter_count() method to tally all weights and biases. Furthermore, shape() reports the complete layer dimensions. 
classify(input) runs a forward pass and returns the neuron with max output. We also incorporated file I/O functionality through the write_to_file 
method. The function essentially encodes the number of layers, size, activations, etc.  



\subsubsection{Convolutional neural networks}
Fully-connected neural networks are very general, but they do not take advantage of any potential structure of the input to minimize the number of required parameters or improve efficiency or performance.
Enter, convolutional neural networks (CNNs).
CNNs, primarily designed for image recognization and classification, use the spatial structure of an image to allow for shared weights and biases, reducing the number of required parameters.
In particular, CNNs scan a series of filters across the image, each of which conceptually is intended to identify a certain local feature of the image (e.g. identify straight or curved lines).
This scanning process is specifically done by convolving each filter with the image, hence the `convolutional' in CNN.
As described in \textcolor{blue}{\autocite{Making_faster}}, this convolution step can be implemented efficiently via matrix multiplication by first reshaping the input matrices and filters.
Although we do not give the explicit formulas here (see \textcolor{blue}{\autocite{Solai_2018}}), the derivatives of the output of a convolution layer with respect to the inputs and filters can also be computed via a convolution.

CNNs often also contain pooling layers, which further reduce the dimensionality of the data.
These layers scan a `window' along the input matrix, combining all the elements in each window into a single output element.
This combination process can be done in a number of ways, including taking the maximum, average, or sum of the window elements, which are referred to as max pooling, average pooling, and sum pooling, respectively.
These pooling layers do not have any parameters, and their outputs are simply related to the elements of each window, so the derivatives of this layer are straightforward to compute.

A CNN usually involves a sequence of several of these convolution and pooling layers.
However, once the dimensionality of the input data has been sufficiently reduced, the final classification is usually performed by a fully-connected neural network, which constitutues the last few layers of the CNN.
As with the fully-connected NN, training proceeds by iteratively passing the training inputs through each layer to get the model output, and then moving back through the layers via the chain rule to compute the loss gradient.

\begin{center}
 \textbf{Implementation:}
\end{center}

Following the specifications above, the Convolutional Neural Network implementation within ML_kit implements a sequence of three kinds of layers. 
We define a type enum Layer to consist of the convolutional, pooling, and fully connected layers. 

We implement convolutional layers $ConvLayer$ that store a bank of multi-depth filters with a per filter bias. Furthermore, we included hyperparameters 
such as stride and padding. Pooling Layers implementex max, avg, or sum pooling and include the forward propagation abd backprop functionality in the feedforward 
and backprop methods respectively. We implement fully connected layeres $FullLayer$ as discussed above within the Fully Connected Neural Network section. 

These layers are wrapped in our CNN datastructure $ConvNeuralNet$ which holds the subsequent methods. The compute_final_layaer() method sequentially calls the 
feed_forward method of each respective layer, returning final activations. The populate_gradients() method does a forward pass to record activations, computes 
each intial gradient, and backtracks via calling each layers back_prop function, reshaping when needed. Grad descent step calls each layers update_params() method. 
sgd_batch_step() essentially does a forward and backward pass given the functions above and updates layers accordingly. train_sgd() loops over user defined epoch and applies 
agd_batch_step(). 


\subsubsection{Stochastic gradient descent}
The training of these neural networks is often powered by stochastic gradient descent (SGD).
SGD is a scheme for updating the paraemters of a network using the gradient of the loss for small sampled batches of training examples.
In particular, to perform SGD, a set of training examples is provided, along with a specification of batch size and number of epochs.
In each epoch, the whole set of training examples is sampled into different smaller sets specified by the batch size.
The network parameters are then updated based on the loss incurred by each batch of training examples, rather than the entire training set.
This greatly reduces the computational cost in each update step, while not changing the performance of the model in expectation.
Each batch step specifically involves iterating through each training example in the batch.
Each training input is passed forward through the layers of the network to compute the output and loss, which is then passed back through the layers to compute the the derivative of the loss with respect to each parameter via the chain rule.
These derivatives are summed for each example from the batch, the resulting gradient is normalized, and then each parameter $p$ is updated as according to
\begin{equation*}
  \begin{aligned}
    p \mapsto p - \alpha\frac{\partial L}{\partial p} ~,
  \end{aligned}
\end{equation*}
where $\frac{\partial L}{\partial p}$ is the normalized derivative of the loss $L$ with respect to parameter $p$ and $\alpha$ is the step size, or `learning rate'.
This process is repeated for each batch in the training set, yielding an epoch.
Many epochs are usually required to train a network to achieve reasonable classification accuracies.

\subsubsection{What we implemented}










\subsection{Principle Component Analysis and SVD}

Singular Value Decomposition (SVD) is a factorization of a matrix $A \in \R^{m \times n}$ into
\[
    A = U \Sigma V^\top
\]
where $U \in \R^{m \times m}$ and $V \in \R^{n \times n}$ are orthogonal matrices with 
columns $\vect u_1, \ldots \vect u_m$ and $\vect v_1, \ldots, \vect v_n$ respectively, and $\Sigma$
is a diagonal matrix of the form (assuming $n \leq m$)
\[
    \Sigma = \left[
        \begin{array}{cccc}
            \sigma_1 & & & \\
            & \sigma_2 & & \\
            & & \ddots & \\
            & & & \sigma_n \\
            & & & \\
            & & & 
        \end{array}
    \right] \in \R^{m \times n}
\]
where $\sigma_1 \geq \sigma_2 \geq \cdots \geq \sigma_n \geq 0$ are the \emph{singular values} of $A$.
This allows us to write $A$ as the linear combination
\[
    A = \sum_{i = 1}^n \sigma_i \vect{u}_i \vect{v}_i^\top
\]
Note that because the singular values are sorted in decreasing order, we can effectively ``save data'' in 
representing $A$ by truncating all but the first $r \leq n$ singular vectors, as the last items in the sum do not 
contribute as much to the 
overall product. The immediate application is identifying the most significant axes of correlation in data, allowing 
dimensionality reduction of datasets by re-writing each item in the basis $\vect{v}_1, \ldots, \vect{v}_r$.
Commonly, the ``data'' of concern is written into the rows of $A$.

\subsubsection{What we implemented}

Our library implements the Golub-Kahan SVD algorithm (described in ``Matrix Computations,'' by Golub and van Loan.
\textcolor{blue}{\autocite{golub13}}) which begins by bi-diagonalizing $A$, then performing SVD on the bidiagonalization,
as the numerical stability and performance are better. Once we were able to compute the SVD of any matrix, we could use 
SVD as a subroutine in other useful techniques.

The (subjectively) coolest application of SVD we implemented is image compression. By taking an $m \times n$ image and 
writing it as a 4-tuple of matrices $(R, G, B, A)$ representing the red, green, blue, and alpha channels of the pixels,
we can perform SVD on each color channel, store only the SVD representation after truncating insignificant singular
vectors, and then de-compress the image later by computing $U \Sigma V^\top$ for each color channel. In practice,
one can discard roughly half the singular values and still obtain a recognizeable image. Examples of this and discussion 
of runtime and compression rates will be left to the evaluation section.

Another common goal in statistics is finding the ``line of best fit'' of a set of points, or in higher dimensions,
a $k$-dimensional plane of best fit. For a cluster of data centered at the origin, the first singular vector, $\vect v_1$,
is the line of best fit. This is because SVD is equivalently defined as an optimization proceedure where 
\[
    \vect v_1 = \argmax_{\|x\| = 1} \|Ax\|
\]
Since $\vect v_1$ is maximizing the sum of squares of the inner products with all the data points, it is minimizing 
the sum of squared distances from each data point to the line spanned by $\vect v_1$. More generally, taking the first 
$\vect v_1, \ldots, \vect v_k$ vectors, we obtain a basis for a $k$-dimensional ``plane of best fit.'' 

We implement a proceedure which takes a data matrix $D \in \R^{n \times m}$ whose columns are each $\R^n$ data points, 
and returns $\vect \mu \in \R^n$ and $V \in \R^{n \times k}$ such that the plane defined by 
\[
    \left\{\vect \mu + \sum_{i = 1}^{k} \alpha_i \vect v_i \mid \alpha_i \in \R\right\}
\]
is the plane of best fit for the data. This was a very minor part of our library and only consisted 
of a few extra lines of code, but since we had the functionality to easily implement it, we figured why not?

\subsection{Relationship to Other Work}

Linear algebra is the foundatin to machine learning, so we needed to use a good linear algebra library. Sylvan had 
previously spent winter break working on \texttt{matrix\_kit}, which is a pure-Rust linear algebra package that 
implemented incredibly basic matrix-vector operations. We continued developing this library in parallel with 
\texttt{ml\_kit} over the semester as we recognized more features that were needed from the linear algebra library. 
So, the sum of our work for the course can be thought of as the entirety of \texttt{ml\_kit}, as well as significant
improvement to the functionality of \texttt{matrix\_kit}.

The \texttt{matrix\_kit} library can be found on GitHub at \url{https://github.com/SylvanM/matrix_kit}.

\section{Evaluation}

\subsection{Neural Networks}
Neural network based learning (i.e. deep learning) is a type of supervised learning that involves using black-box models which depend on a large number of tunable parameters to classify complicated inputs.
The models are trained by iteratively evaluating their performance on training examples and updating their parameters accordingly, usually through gradient descent.
We call these black-box models `networks' because they are generally conceptualized as a series of composed functions, or `layers', where each layer linearly maps some input vector (or, more generally, tensor) to a possibly differently-sized output vector, and then applies a simple element-wise non-linear `activation function'.
This sequence of highly inter-connected linear functions interspersed with element-wise non-linearities enables the model to identify complex features only using parameters that have relatively simple and computable gradients, ensuring efficient training.

\subsubsection{Fully-connected neural networks}
The traditional neutral network (NN) is a series of fully-connected layers, meaning each input element of a layer has a parameter through which it can linearly affect each output element.
In particular, the function associated with each layer is of the form \textcolor{blue}{\autocite{Nielsen_2015}}
\begin{equation*}
  \begin{aligned}
    \vect{v}_{\text{out}} = f(\vect{v}_{\text{in}}) = \sigma\left(\mathbf{W}\vect{v}_{\text{in}} + \vect{b}\right) ~,
  \end{aligned}
\end{equation*}
where $\mathbf{W}$ is a matrix of `weight' parameters, $\vect{b}$ is a vector of `bias' parameters, and $\sigma$ is the non-linear activiation function.
The derivative of the output vector with respect to the input vector and these parameters can then be simply calculated as
\begin{equation*}
  \begin{aligned}
    \frac{\partial v_{\text{out}, i}}{\partial W_{i, j}} = v_{\text{in}, j} \cdot \sigma'\left(\vect{x}\right)\hspace{3pt}, \hspace{30pt} \frac{\partial v_{\text{out}, i}}{\partial b_{i}} & = \sigma'\left(\vect{x}\right)\hspace{3pt},\text{ and} \hspace{23pt} \frac{\partial v_{\text{out}, i}}{\partial v_{\text{in}, j}} = W_{i, j} \cdot \sigma'\left(\vect{x}\right) ~,
  \end{aligned}
\end{equation*}
where $\vect{x} = \mathbf{W}\vect{v}_{\text{in}} + \vect{b}$ and all unspecified derivatives are zero.
Thus, the output of the entire neural-network model can be simply computed as the composition of each layer function, and the gradient of the loss with respect to the model parameters (to use for parameter updates) can be determined by iteratively applying the chain rule to the above derivatives.

\begin{center}
 \textbf{Implementation:}
\end{center}

The NeuralNetwork implementation within ML_kit is straightforward and exncapsulates everything needed to build, inspect, and run a feed-forward network.
Initially, a NeuralNet holds three parallel vectors: 

1. Weight Matrices $Vec<Matrix<f64>>$, each matrix represents the weight between two layers. 
2. Bias Vectors, one per non-input layer. Each is stored in a column sized vector of size $m * 1$ where m is the number of neurons in said layer. 
pre-activation vector prior to applying the activation function. 
3. Activation Functions, a vector of activations indicate which activation is to be used upon each layer. Sigmoid, Relu, etc. 

Our NeuralNet construction allows for two ways of instantiation. You can call NeuralNet::new(weights, biases, activations) 
to instantiate a network of user provided parameters. Otherwise, for ease of testing, you can call NeuralNet::from_shape(shape, activations)
to create a zero initialized network and fill with NeuralNet::random_network(shape, activations). 

Once instantiated, our compute_final_layer(input) function incorporates the forward pass logic. Starting with a column vector, for each layer 
$l$ we multiply with weight matrix $W_l$ and add bias $b_l$, and apply the element wise activation. The resulting column is then 
carried into the next layer until we return the output vector. We also incorporate functions such as compute_raw_layers() and compute_raw_and_full_layers(). 

We also implemented a paramter_count() method to tally all weights and biases. Furthermore, shape() reports the complete layer dimensions. 
classify(input) runs a forward pass and returns the neuron with max output. We also incorporated file I/O functionality through the write_to_file 
method. The function essentially encodes the number of layers, size, activations, etc.  



\subsubsection{Convolutional neural networks}
Fully-connected neural networks are very general, but they do not take advantage of any potential structure of the input to minimize the number of required parameters or improve efficiency or performance.
Enter, convolutional neural networks (CNNs).
CNNs, primarily designed for image recognization and classification, use the spatial structure of an image to allow for shared weights and biases, reducing the number of required parameters.
In particular, CNNs scan a series of filters across the image, each of which conceptually is intended to identify a certain local feature of the image (e.g. identify straight or curved lines).
This scanning process is specifically done by convolving each filter with the image, hence the `convolutional' in CNN.
As described in \textcolor{blue}{\autocite{Making_faster}}, this convolution step can be implemented efficiently via matrix multiplication by first reshaping the input matrices and filters.
Although we do not give the explicit formulas here (see \textcolor{blue}{\autocite{Solai_2018}}), the derivatives of the output of a convolution layer with respect to the inputs and filters can also be computed via a convolution.

CNNs often also contain pooling layers, which further reduce the dimensionality of the data.
These layers scan a `window' along the input matrix, combining all the elements in each window into a single output element.
This combination process can be done in a number of ways, including taking the maximum, average, or sum of the window elements, which are referred to as max pooling, average pooling, and sum pooling, respectively.
These pooling layers do not have any parameters, and their outputs are simply related to the elements of each window, so the derivatives of this layer are straightforward to compute.

A CNN usually involves a sequence of several of these convolution and pooling layers.
However, once the dimensionality of the input data has been sufficiently reduced, the final classification is usually performed by a fully-connected neural network, which constitutues the last few layers of the CNN.
As with the fully-connected NN, training proceeds by iteratively passing the training inputs through each layer to get the model output, and then moving back through the layers via the chain rule to compute the loss gradient.

\begin{center}
 \textbf{Implementation:}
\end{center}

Following the specifications above, the Convolutional Neural Network implementation within ML_kit implements a sequence of three kinds of layers. 
We define a type enum Layer to consist of the convolutional, pooling, and fully connected layers. 

We implement convolutional layers $ConvLayer$ that store a bank of multi-depth filters with a per filter bias. Furthermore, we included hyperparameters 
such as stride and padding. Pooling Layers implementex max, avg, or sum pooling and include the forward propagation abd backprop functionality in the feedforward 
and backprop methods respectively. We implement fully connected layeres $FullLayer$ as discussed above within the Fully Connected Neural Network section. 

These layers are wrapped in our CNN datastructure $ConvNeuralNet$ which holds the subsequent methods. The compute_final_layaer() method sequentially calls the 
feed_forward method of each respective layer, returning final activations. The populate_gradients() method does a forward pass to record activations, computes 
each intial gradient, and backtracks via calling each layers back_prop function, reshaping when needed. Grad descent step calls each layers update_params() method. 
sgd_batch_step() essentially does a forward and backward pass given the functions above and updates layers accordingly. train_sgd() loops over user defined epoch and applies 
agd_batch_step(). 


\subsubsection{Stochastic gradient descent}
The training of these neural networks is often powered by stochastic gradient descent (SGD).
SGD is a scheme for updating the paraemters of a network using the gradient of the loss for small sampled batches of training examples.
In particular, to perform SGD, a set of training examples is provided, along with a specification of batch size and number of epochs.
In each epoch, the whole set of training examples is sampled into different smaller sets specified by the batch size.
The network parameters are then updated based on the loss incurred by each batch of training examples, rather than the entire training set.
This greatly reduces the computational cost in each update step, while not changing the performance of the model in expectation.
Each batch step specifically involves iterating through each training example in the batch.
Each training input is passed forward through the layers of the network to compute the output and loss, which is then passed back through the layers to compute the the derivative of the loss with respect to each parameter via the chain rule.
These derivatives are summed for each example from the batch, the resulting gradient is normalized, and then each parameter $p$ is updated as according to
\begin{equation*}
  \begin{aligned}
    p \mapsto p - \alpha\frac{\partial L}{\partial p} ~,
  \end{aligned}
\end{equation*}
where $\frac{\partial L}{\partial p}$ is the normalized derivative of the loss $L$ with respect to parameter $p$ and $\alpha$ is the step size, or `learning rate'.
This process is repeated for each batch in the training set, yielding an epoch.
Many epochs are usually required to train a network to achieve reasonable classification accuracies.

\subsubsection{What we implemented}










\subsection{Principle Component Analysis and SVD}

Singular Value Decomposition (SVD) is a factorization of a matrix $A \in \R^{m \times n}$ into
\[
    A = U \Sigma V^\top
\]
where $U \in \R^{m \times m}$ and $V \in \R^{n \times n}$ are orthogonal matrices with 
columns $\vect u_1, \ldots \vect u_m$ and $\vect v_1, \ldots, \vect v_n$ respectively, and $\Sigma$
is a diagonal matrix of the form (assuming $n \leq m$)
\[
    \Sigma = \left[
        \begin{array}{cccc}
            \sigma_1 & & & \\
            & \sigma_2 & & \\
            & & \ddots & \\
            & & & \sigma_n \\
            & & & \\
            & & & 
        \end{array}
    \right] \in \R^{m \times n}
\]
where $\sigma_1 \geq \sigma_2 \geq \cdots \geq \sigma_n \geq 0$ are the \emph{singular values} of $A$.
This allows us to write $A$ as the linear combination
\[
    A = \sum_{i = 1}^n \sigma_i \vect{u}_i \vect{v}_i^\top
\]
Note that because the singular values are sorted in decreasing order, we can effectively ``save data'' in 
representing $A$ by truncating all but the first $r \leq n$ singular vectors, as the last items in the sum do not 
contribute as much to the 
overall product. The immediate application is identifying the most significant axes of correlation in data, allowing 
dimensionality reduction of datasets by re-writing each item in the basis $\vect{v}_1, \ldots, \vect{v}_r$.
Commonly, the ``data'' of concern is written into the rows of $A$.

\subsubsection{What we implemented}

Our library implements the Golub-Kahan SVD algorithm (described in ``Matrix Computations,'' by Golub and van Loan.
\textcolor{blue}{\autocite{golub13}}) which begins by bi-diagonalizing $A$, then performing SVD on the bidiagonalization,
as the numerical stability and performance are better. Once we were able to compute the SVD of any matrix, we could use 
SVD as a subroutine in other useful techniques.

The (subjectively) coolest application of SVD we implemented is image compression. By taking an $m \times n$ image and 
writing it as a 4-tuple of matrices $(R, G, B, A)$ representing the red, green, blue, and alpha channels of the pixels,
we can perform SVD on each color channel, store only the SVD representation after truncating insignificant singular
vectors, and then de-compress the image later by computing $U \Sigma V^\top$ for each color channel. In practice,
one can discard roughly half the singular values and still obtain a recognizeable image. Examples of this and discussion 
of runtime and compression rates will be left to the evaluation section.

Another common goal in statistics is finding the ``line of best fit'' of a set of points, or in higher dimensions,
a $k$-dimensional plane of best fit. For a cluster of data centered at the origin, the first singular vector, $\vect v_1$,
is the line of best fit. This is because SVD is equivalently defined as an optimization proceedure where 
\[
    \vect v_1 = \argmax_{\|x\| = 1} \|Ax\|
\]
Since $\vect v_1$ is maximizing the sum of squares of the inner products with all the data points, it is minimizing 
the sum of squared distances from each data point to the line spanned by $\vect v_1$. More generally, taking the first 
$\vect v_1, \ldots, \vect v_k$ vectors, we obtain a basis for a $k$-dimensional ``plane of best fit.'' 

We implement a proceedure which takes a data matrix $D \in \R^{n \times m}$ whose columns are each $\R^n$ data points, 
and returns $\vect \mu \in \R^n$ and $V \in \R^{n \times k}$ such that the plane defined by 
\[
    \left\{\vect \mu + \sum_{i = 1}^{k} \alpha_i \vect v_i \mid \alpha_i \in \R\right\}
\]
is the plane of best fit for the data. This was a very minor part of our library and only consisted 
of a few extra lines of code, but since we had the functionality to easily implement it, we figured why not?

\printbibliography

\end{document}